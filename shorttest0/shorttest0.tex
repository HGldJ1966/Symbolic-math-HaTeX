\documentclass{article}\usepackage[utf8]{inputenc}\usepackage{amsmath}\author{Justus Sagemüller}\title{Simple example}\begin{document}\maketitle{}\section{Hello}This is a simple example using the \mbox{H\hspace{-0.3ex}\textsc{a}\hspace{-0.3ex}\TeX{}} library and some math stuff. \subsection{Arithmetics with infix operators}\[
{{4} ^ {{\left({2} ^ {3}\right)} ^ {2}}} - {{{{10000} \cdot {10000}} \cdot {\left({10000} \cdot {10000}\right)}} \cdot {\left({{10000} \cdot {10000}} \cdot {10000}\right)}}
\] is ${3.40282} \cdot {{10} ^ {38}}$. For $x=19$ and $\tau{}={2} \cdot {\pi{}}$, \begin{align*}{{2} + {{7} \cdot {\left({6} - {\tau{}}\right)}}} - {e^{{5} - {\sqrt{{{x} ^ {2}} + {\frac{4}{\pi{}}}}}}}
   &\approx{}{1.7702} \cdot {{10} ^ {-2}}.\end{align*}\subsection{Simple finite sums}\begin{align*}\sum_{{n\in{}\left\{0,1,4,5\right\}}}{\tfrac{5}{2}} - {n}
   &=0\end{align*}\begin{align*}\sum_{{n=1}}^{{4}}{\tfrac{5}{2}} - {n}
   &=0\end{align*}\begin{align*}\sum_{{j=1}}^{{40}}\cos{}{\left({\frac{{2} \cdot {\pi{}}}{40}} \cdot {j}\right)}
   &\approx{}{-2.6645} \cdot {{10} ^ {-15}}\end{align*}\begin{align*}{2} \cdot {\sum_{{i=1}}^{{6}}{{i} ^ {2}} + {i}}
   &=224\end{align*}\begin{align*}{\left(\sum_{{i=1}}^{{6}}{{i} ^ {2}} + {i}\right)} \cdot {2}
   &=224\end{align*}\begin{align*}{\left(\sum_{{i=1}}^{{6}}{{i} ^ {2}} + {i}\right)} + {2}
   &=114\end{align*}\subsection{Checking some simple identities}\[
\arcsin{}{\left(\sin{}{\left(\arccos{}{\left(\cos{}{\left(\arctan{}{\left(\tan{}\:{0}\right)}\right)}\right)}\right)}\right)}
\] is $0$, \[
\mathrm{arcsinh}{\left(\sinh{}{\left(\mathrm{arccosh}{\left(\frac{\cosh{}{\left(\mathrm{arctanh}{\left(\tanh{}\:{0}\right)}\right)}}{2}\right)}\right)}\right)}
\]\textbf{ is not. }(Test passed.)\\A simple equations chain:\begin{align*}{10} ^ {18}
   &={{10} ^ {9}} \cdot {{10} ^ {9}}
 \\ &={{{10} ^ {{3} ^ {2}}} \cdot {{10} ^ {5}}} \cdot {{10} ^ {4}}
 \\ &=1000000000000000000.\end{align*}(Test passed.)\\Another equations chain, this time using floats:\begin{align*}{10} ^ {-\:{18}}
   &={{10} ^ {-\:{9}}} \cdot {{10} ^ {-\:{9}}}
 \\ &={{{10} ^ {-{\left({3} ^ {2}\right)}}} \cdot {{10} ^ {-\:{5}}}} \cdot {{10} ^ {-\:{4}}}
 \\ &=\frac{1}{1000000000000000000}.\end{align*}\textbf{Test failed. }Even true mathematical identities may not show to hold when using floating-point arithmetics.\\Equation-chains can also be approximate (``rough''):\begin{align*}{10} ^ {-\:{18}}
   &\approx{}{{10} ^ {-\:{9}}} \cdot {{10} ^ {-\:{9}}}
 \\ &\approx{}{{{10} ^ {-{\left({3} ^ {2}\right)}}} \cdot {{10} ^ {-\:{5}}}} \cdot {{10} ^ {-\:{4}}}
 \\ &\approx{}\frac{1}{999236372934876337}.\end{align*}(Test passed.)\end{document}